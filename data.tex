\documentclass[a4paper,12pt]{article} % добавить leqno в [] для нумерации слева
\usepackage[a4paper,top=1.3cm,bottom=2cm,left=1.5cm,right=1.5cm,marginparwidth=0.75cm]{geometry}
%%% Работа с русским языком
\usepackage{cmap}					% поиск в PDF
\usepackage{mathtext} 				% русские буквы в фомулах
\usepackage[T2A]{fontenc}			% кодировка
\usepackage[utf8]{inputenc}			% кодировка исходного текста
\usepackage[english,russian]{babel}	% локализация и переносы
\usepackage[16pt]{extsizes}

\usepackage{graphicx}

\usepackage{wrapfig}
\usepackage{tabularx}

\usepackage{hyperref}
\usepackage[rgb]{xcolor}
\hypersetup{
colorlinks=true,urlcolor=blue
}

%%% Дополнительная работа с математикой
\usepackage{amsmath,amsfonts,amssymb,amsthm,mathtools} % AMS
\usepackage{icomma} % "Умная" запятая: $0,2$ --- число, $0, 2$ --- перечисление

%% Номера формул
\mathtoolsset{showonlyrefs=true} % Показывать номера только у тех формул, на которые есть \eqref{} в тексте.

%% Шрифты
\usepackage{euscript}	 % Шрифт Евклид
\usepackage{mathrsfs} % Красивый матшрифт

%% Свои команды
\DeclareMathOperator{\sgn}{\mathop{sgn}}

%% Перенос знаков в формулах (по Львовскому)
\newcommand*{\hm}[1]{#1\nobreak\discretionary{}
{\hbox{$\mathsurround=0pt #1$}}{}}

%%% Заголовок
\author{Вихлянцев Константин Игоревич}
\title{Лабораторная работа №1.1.1

Дифференцирование.
}
\date{\today}

\begin{document}

\begin{titlepage}
	\begin{center}
		{\large МОСКОВСКИЙ ФИЗИКО-ТЕХНИЧЕСКИЙ ИНСТИТУТ (НАЦИОНАЛЬНЫЙ ИССЛЕДОВАТЕЛЬСКИЙ УНИВЕРСИТЕТ)}
	\end{center}
	\begin{center}
		{\large Физтех-школа кибернетики и радиотехники}
	\end{center}
	
	
	\vspace{1cm}
	{\huge
		\begin{center}
			{\bf Отчёт о выполнении лабораторной работы по матану 1.1.1}\\
			Дифференцирование.
		\end{center}
	}
	\vspace{1cm}
	\begin{flushright}
		{\LARGE Автор:\\ Вихлянцев Константин Игоревич \\
			\vspace{0.2cm}
			Б01-103}
	\end{flushright}
	\vspace{1cm}
	\begin{center}
		Долгопрудный 2021
	\end{center}
\end{titlepage}

1. Выражение для дифференцирования:

\[x^{2}\]

2. Первоначальная обработка выражения:

\[x^{2}\]

3. Выражение после дифференцирования:

\[{2}\cdot{{x^{(2-1)}}\cdot{1}}\]

4. Упрощенное выражение после дифференцирования:

\[{2}\cdot{x}\]

\end{document}
